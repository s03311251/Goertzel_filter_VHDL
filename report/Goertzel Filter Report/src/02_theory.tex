\section{Goertzel Algorithm Theory}\label{sec:theory}
As mentioned above, the main purpose of the Goertzel Algorihm is to detect one selectable frequency component from a discrete signal (\cite{Chen1998ModifiedGA}). To describe how this is achieved, the Goertzel algorithm is often understood to have the form of a digital filter, therefore it is often referred to as \textit{Goertzel Filter}\footnote{For the purpose of this report, both names will be used interchangeably} as well. The core of the algorithm is depicted in Fig. n. \ref{fig:gf_overview}. As suggested by Regnacq L. et al in\cite{9464344}, such filter can be observed to have two main parts, namely a main part which resembles a second-order infinite impulse response (IIR) filter, and a second part which can be seen as a feed-forward path using one complex multiplication.\\
The purpose of the IIR part is to calculate an intermediate sequence $s(n)$ from a $N$ sample digital signal $x(n)$ according to the following formula:
\begin{equation}
    s(n) = x(n) + 2cos(\omega_0)s(n-1) - s(n-2)
    \label{eq:UPDATE}
\end{equation}
where $\omega_0 = \frac{2\pi m}{N}$. In this case, $m$ refers to the frequency index, or bin, to scout.\\
The second stage of the filter, finally, can be seen as a finite impulse response (FIR), since it does not use any of the past outputs and produces an outputs sequence $y(n)$ applying the following filter:
\begin{equation} \label{eq:FIR}
    y(n) = s(n) - e^{-j\omega_0}s(n-1)
\end{equation}
The IIR part of the Goertzel algorithm is performed $N$ times iteratively, while the FIR part only once, namely for $s(N)$. Some authors point out that such calculation should be performed periodically, i.e. after a $P$ number of samples and not only once. For example, Dulik T. in \cite{dulik} suggests that such period depends on the application demands, e.g. a DTMF would be triggered with a period $P$ being $100<P<300$. However, as indicated in the same paper, such periodic calculation in FIR part should be performed only if an exact DFT magnitude of the frequency is required, as it would improve the accuracy. In the context of frequency detection, the calculation in the FIR part can be skipped or simplified.  \\
Due to the iterative nature of the Goertzel filter, it can be proven to be of a higher order of complexity than the fast Fourier Transform (FFT) in covering the entire frequency spectrum. As a matter of fact, the iterative part of the Goertzel filter possess an asymptotic complexity of order O($N$) per calculated $m$, meaning it must be repeated $k$ times for $k$ frequency components to be calculated. Additionally, for each of the frequency, the algorithm calculates the output sequence $P<N$ times per each frequency considered. For the purpose of this report, we are focusing on frequency detection, therefore we will not consider this as a periodic calculation, i.e. we will assume a period $P=1$. It is therefore trivial to prove that for the calculation of an entire spectrum such complexity approaches O($N^2$) \footnote{The $P$ in O($N^2+ P$) is omitted as we are focusing on frequency detection and only considering the asymptotic behavior}, which is asymptotically much higher than the O($N log_2N$) of the FFT (\cite{FFT}). \\
Although at an obvious disadvantage for a full spectrum calculation, the iterative nature of the Goertzel filter makes it a much more efficient approach for a small set of frequencies, i.e. a set of frequencies of cardinality $M$ with $ M<< N$ , which leaves the asymptotic complexity at approximately O($MN$) $\approx$ O($N$), with $M$ often being simply one. Moreover, the iterative nature makes such filter a perfect candidate for exploiting computations in a parallel fashion (\cite{Chen1998ModifiedGA}) and it helps addressing one limitation of the FFT. As a matter of fact, the FFT requires the number of samples being a power of 2, being it a divide-and-conquer algorithm, while the Goertzel filter does not. This characteristc makes the Goertzel algorithm easier to scale than the FFT.  \\
In addition, the rather simplistic nature of the Goertzel algorithm, which requires only one multiplication and two addition per iteration, makes it numerically more efficient than various FFT implementations, like the one proposed in \cite{Press2007}.
Furthermore, the nature of the algorithm makes it also more memory efficient for $M$ frequencies detection than the FFT algorithm, as it requires only 3$M$ coefficients to be saved. \\
Due to its characteristics, therefore, the Goertzel Filter is also a perfect candidate for small processors and embedded applications. 